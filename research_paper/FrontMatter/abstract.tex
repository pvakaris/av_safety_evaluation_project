The vision of a civilisation where driverless cars seamlessly navigate the lively streets and high-speed underground tunnels, transporting passengers from point A to point B, is remarkably portrayed in many futuristic movies and science fiction novels. However, this futuristic dream is not as far-off as it may seem. In fact, many vehicles operating the streets today already incorporate some level of automation, such as cruise control and automatic parking. While it is clear that we are making progress towards fully autonomous vehicles, the question remains: when will we finally witness entirely driverless cars on the roads? Although autonomous vehicles already hold the potential to surpass human drivers in terms of safety, there is currently no practical way to test this claim. Or is there?
This paper proposes an innovative approach - evaluating high-level vehicle safety through simulations on virtual roads in an automated and time-efficient manner. The proposed software consists of three main components: an automated scenario generation tool that, with the help of a sophisticated machine learning algorithm, can generate diverse driving scenarios, a vehicle safety monitoring system designed for use in the simulator CARLA and data analysis tools allowing for efficient and visualised analysis of obtained data through diagrams and location marking on 2D map representations. In addition to introducing the automated safety testing software bundle, this article also dives deeper into the realm of AVs by discussing the technology, safety vulnerabilities, how AVs are classified and the comparison of human drivers to computer-controlled cars.