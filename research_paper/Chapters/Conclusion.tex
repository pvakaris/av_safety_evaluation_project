\chapter{Conclusion} \label{chap:ten}
This chapter concludes the discussion of AVs and the introduction of an alternative way of measuring vehicle safety in a simulated environment. In \autoref{sect-10.1}, an overview of the project is given, explaining what has been developed. The following section discusses the research relevance of the project and talks about the attitude with which the project was implemented. In \autoref{sect-10.3}, a list of possible future improvements is provided for people willing to take this project further. In \autoref{sect-10.4}, ethical concerns surrounding the project and the autonomous driving field are considered. The chapter is concluded in \autoref{sect-10.5}, after the author's self-reflection about the challenges faced over the development of this project.

\section{What has been presented in this article} \label{sect-10.1}
In the early chapters of this article, it was established that for autonomous vehicles to appear on the streets, their safety needs to be recognised by the general public. However, we soon discovered that traditional ways of testing autonomous vehicles, namely by exploiting them on the roads, would be inefficient given the time needed and the non-deterministic nature of the world. A new and innovative approach was needed - an approach that would look at the problem through another perspective - the simulated one. The article focused on developing a software system allowing for automated vehicle safety assessment in the virtual setting. 

Built around the simulator CARLA, the introduced system consists of three main components. The first is an automatic scenario generation algorithm that, with the help of a sophisticated machine learning algorithm and a path-building script, can create diverse traffic scenarios. Generated scenarios allowed for a vehicle to be tested on the roads and its actions recorded by the safety monitoring system, which is the second component. The system is responsible for observing the vehicle's actions and transmitting the data to its monitors that record the violations in the database. The data analysis tool then handles the recorded data by generalising it and preparing it for visualisations using diagrams and 2D map representations, allowing for further analysis of behaviour patterns. In addition, an equation for evaluating the vehicle's performance in a given scenario was developed that allows for a simulation run to be allocated a safety performance score. Moreover, besides all the technical details involving the development of the system, the article also discussed in detail how the ideas of autonomous vehicles matured throughout human history, compared the AI behaviour with that of the human drivers, tested the proposed software system with participants and evaluated how well the system works in practice.

\section{Future improvements} \label{sect-10.2}
As with any bigger project developed in a limited amount of time and with limited resources, there is always a place for improvement. This section aims to give some guidelines about what could be improved and in what direction.

Let us start with the scenario generation algorithm. As mentioned in the \autoref{sect-5.4}, the algorithm could be upgraded in several ways. Firstly, a set of real-world data could be used to have a more accurately trained algorithm whose generated scenarios would be more reflective of the real-world conditions. In addition, a more sophisticated algorithm could be used able to create even more diverse and unique situations, allowing for the automatic generation of other vehicle behaviour. For instance, generating situations where an actor in the simulation acts according to some patterns when the participant vehicle is close (suddenly changing lanes, for instance). However, the current implementation is only able to create general scenarios where the behaviour is set to all the vehicles with the hope that some interesting and unusual traffic conditions will make their way to the surface. What would also be good to have is a system able to analyse how challenging and diverse the scenarios actually are.

The path generation algorithm also has places for improvement, starting with how the paths are built. At the moment, the algorithm creates paths by trial and error, randomly choosing the starting element and then proceeding from there. It would be much better if it did that in a structured way, reasoning about why the path has to look like that and relating it to the generated scenario. This would allow for even more exciting paths and behaviour patterns to be built.

In terms of the safety monitoring system and the data analysis tools, they currently do not provide a way to use the observations for model improvement. They are there to record the violations and then look at the general behaviour patterns without linking the observations with some mistakes in the model. This, in addition to the improvements to the collision sensor (mentioned in \autoref{sect-9.3}, should be addressed in future updates.

Moreover, the current software does not provide automated tests to check the components' validity and correctness. As with autonomous vehicles, the systems measuring their safety must also be flawless. For that, a comprehensive test suite should be developed, ensuring that all components work as intended and that updates to one do not break anything in another.

\section{Research relevance and project's integrity} \label{sect-10.3}

This article aims to contribute to the field of autonomous driving by presenting a method of testing driverless cars in realistic traffic situations. As highlighted earlier in the paper, demonstrating the safety and reliability of autonomous vehicles to the general public is vital in bringing all the hard work in the field to reality. This statement is supported by other literature as well \cite{zhang2019roles}. Every little step in this direction is significant, and by raising people's curiosity and awareness about the latest developments, we are getting closer to achieving this goal. By introducing this project, we hope to grab the attention of young people and inspire their engagement in this field, emphasizing that it can be both enjoyable and fruitful.

In addition to contributing to the field of autonomous driving, the project aimed to leverage existing open-source projects. By doing so, the project aimed to both benefit from existing systems and introduce innovative ways of utilizing these tools while also showing appreciation for other software developers and their creations. This article expresses gratitude towards the projects mentioned, namely the CARLA simulator and the CommonRoad Scenario Designer.

All parties involved in the project's development were aware of the BCS Code of Conduct\cite{bcs_code} and always sought to follow the rules. That is why the project was developed with utmost care and honesty, first learning and understanding the problem and only then trying to accomplish the goals. Moreover, the research involving human participants was carried out in a professional and respectful manner resulting in mutual professional and personal growth.

\section{Ethical concerns} \label{sect-10.4}

The field of autonomous driving has always been concerned with ethical questions, as the introduction of such technology would have a significant impact on the lives of many people. Thus, the intentions of engineers need to be carefully considered, and the possible implications rationally weighted. There is no room for doubt regarding the well-being of living beings. Therefore, it is essential to thoroughly test the underlying technology and hardware to ensure they meet the necessary standards. As the use of artificial intelligence in the field is inevitable, proper precautions must be taken. The algorithms cannot be biased, resulting in harm to some groups of people because of the lack of training data or biases in the training data. The systems must be developed with the intention of producing rational and unbiased decisions resulting in the best possible outcome for all parties.

Although this project did not extensively explore the ethical implications of autonomous driving, it did take steps to prevent biased outcomes in the proposed scenario generation method. The article aimed to create diverse scenarios for vehicle safety testing to get comprehensive analysis as discussed in \autoref{chap:two} and \autoref{chap:five}. Additionally, it was assumed that CARLA's set of models for vehicles and pedestrians was diverse; thus, the algorithm randomly selected models without any prejudice. While these steps may not have addressed all potential biases, they demonstrate a commitment to ethical considerations in the field. This project is in favour of ethical and unbiased project development.

\section{Author's self-reflection} \label{sect-10.5}

The purpose of this section is for me, as the author of this paper, to reflect on my journey and the work produced. 

Doing this project allowed me to better understand the complexities surrounding autonomous driving and provided me with hands-on experience contributing to the field. The journey was challenging and full of unexpected changes and issues, constantly pushing me towards upgrading the project and looking for new ways of achieving my goals.
 
Having minimal programming experience in Python and no experience in machine learning, coming up with an automated way of generating the scenarios proved to be particularly challenging. The journey began with studying the machine learning methods on my own, then moved to implementations of various classification and regression algorithms and finally to applying the knowledge in solving the scenario generation in this project. Although this project presents a simple and not the best way of generating the scenarios, attempts to have a more sophisticated algorithm guided by deep neural networks were made. However, the method was not pursued because of the lack of suitable training data and a deep understanding of the CARLA simulator and the advanced machine learning methodologies. In addition, attempts were made to access some global data sources (like the SafetyPool repository\cite{safetypool}) containing scenarios that could be used as training examples. Unfortunately, several tries to get access did not bring any results.

Despite having many technical issues with my computer, which was barely running the simulations at the lowest quality graphics, I managed to successfully conclude my project and develop a prototype software system for vehicle safety evaluation in a virtual environment. I wish the university had provided us with either more lab time or the right equipment to engage with the project fully. Nevertheless, I enjoyed doing this project, and the challenges faced did not discourage but motivated me to move forward and engage in problem-solving.

I believe that the prototype introduced in this article has the potential to become a tool that would be helpful to the industry. However, much more work must be done for it to be effective and trustworthy, and it should be pursued in a team of developers rather than by a single person. Looking back, I think I have taken a bite too big to chew and decided to create an extensive and comprehensive system. I should have chosen one specific aspect, for instance, scenario generation, and focused on doing that only. The end result would have been a well-constructed artefact instead of a broad but less well-implemented system. The possible ways in which the system could be continued were expressed in the previous section.

The journey also taught me about time planning and work organisation. It also showed me that developing software in groups is usually easier than working entirely independently.

In the end, I feel that I have learned many new skills, such as implementing my own machine-learning algorithms, writing shell scripts, performing multi-threading in Python, working with JSON and XML files, analysing data, automating processes and writing academic texts with LaTeX. I also realised that I need to improve my software engineering skills and deepen my knowledge of artificial intelligence to be a better expert in the future and make the field of IT even more influential.