\chapter{The simulated world} \label{chap:four}

In this chapter, we will be focusing on the simulator used in this project - CARLA. The reasons why this simulation framework was chosen, along with its advantages and role in the research, are explained in detail in \autoref{sect-4.1}. Additionally, \autoref{sect-4.2} covers the limitations of the simulator and explores the potential inaccuracies that may arise during the research due to action taking place in the simulated environment.

\section{CARLA simulator} \label{sect-4.1}
As mentioned in the introduction, this work aims to design a system for evaluating vehicle safety performance in realistic simulated driving situations and later evaluate the system's accuracy by testing it with the participants. For this, a sophisticated simulator was needed that could provide a reasonably accurate representation of the human world, from road infrastructure and people to physics and weather impacting them. In addition, it had to have a way of manipulating the world that is being simulated, launching a first-person driving mode, and connecting AV implementations to the simulation. Compared to other competing frameworks, CARLA (Car Learning to Act) \cite{dosovitskiy2017carla} has proven to be an ideal choice for this project.

One of the key benefits of CARLA is its scalable client-server architecture, which allows for efficient manipulation and customization of the simulated environment. The server runs the simulation, renders objects and sensors, and maintains physics. On the other hand, the client can manipulate the simulated world by changing weather conditions, spawning actors and controlling them, and getting information about the simulated world and what is happening there at any given time step. It does so using the CARLA Python API, which provides a comprehensive set of use cases and commands well-documented by CARLA developers. Another benefit of the client-server architecture is that it is relatively easy to connect different AV agent implementations to the simulator and have a defined behaviour of the server and all its components working separately on their own. 

The third benefit is the Agent implementations provided by the CARLA development team. They allow self-driving agents to be put on the map and instructed on where to drive. Although far more sophisticated AI implementations can be bridged with CARLA, like Autoware.ai, which is briefly mentioned in section \autoref{sect-9.1}, the CARLA BasicAgent is used to compete against human drivers in the software testing in \autoref{chap:nine}. The main reason for that is that this project aims to introduce a way of automatically testing the safety of AI implementations operating vehicles rather than connecting different implementations with the simulator. The proposed ideas work with any implementations that can run in the CARLA environment. The safest and most time-efficient way to ensure the framework performs its tasks was to use built-in and easy-to-customise AV implementations

\section{The limitations of the simulator} \label{sect-4.2}
Although CARLA is a powerful tool for testing how AI agents and human drivers operate the vehicles, it is important to understand that it has limitations compared to the real world.

One of them is that it cannot account for the real world's unpredictability. Many factors could impact how an AV reacts, and it is challenging to hard-code them all when designing the scenarios that could occur on the road, similar to how it is difficult to tell the AV how to react in all those situations. There are infinite possible scenarios; in some, the autonomous agent could react well; in others, not so much. This research focuses on designing scenarios with realistic traffic conditions and unpredictable events that require quick reaction and decision-making. The scenarios will test the adaptability and robustness of AVs in unexpected circumstances while also monitoring the overall vehicle's safety.

Another simulator's limitation is the fixed number of road conditions programmed into it. Although CARLA offers a wide variety of customisable stuff, it is still restrained by parameters set by the developers. To exemplify this, snowy winter conditions with iced roads cannot be simulated because there is no way to facilitate this. In addition, there is no efficient way to check if the vehicle has stopped at a stop sign, which is the opposite when talking about checking if the vehicle ran a red light.
 
Moreover, it is important to note that CARLA is only as accurate and good as the algorithms that power it. If the internal calculations are inaccurate, that could affect the simulations and lead to incorrect conclusions about the performance and safety of the drivers.
