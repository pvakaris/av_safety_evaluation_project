\chapter{Assessing vehicle safety} \label{chap:six}
In this chapter, the article proposes a way of assessing vehicle safety on the virtual roads. We begin by giving an overview of the suggested method and explaining why it was chosen in \autoref{sect-6.1}. In \autoref{sect-6.2}, we continue our discussion by explaining what safety metrics are monitored in the simulations and how the collected data is processed. The chapter is finished in \autoref{sect-6.3} after providing a mathematical way of evaluating the vehicle's performance.

\section{Monitoring and recording data} \label{sect-6.1}
After implementing the scenario generation and path-building modules, the next step was to design a sub-system to monitor vehicle actions and record the behaviour, allowing for the data analysis later. As mentioned in \autoref{sect-2.4}, when we talked about the safety of autonomous vehicles, this article deals with evaluating the safety at the vehicle level, meaning how the car acts on the road. For the purpose of this research, it was decided to track the vehicle's actions during the simulation process and then save obtained information to XML files. The XML (Extensible Markup Language) file standard was chosen for its convenient structuring of large files, allowing for efficient data parsing in the tree structures.

The central class of the safety monitoring system is the \textbf{Manager} class, which is responsible for checking the vehicle's state at each time step and informing the other monitoring classes if data is needed to be recorded. The manager class and the monitors are built based on the observer design pattern when one object broadcasts updates to multiple other objects. Having given the overview of the system, let's continue by discussing each component participating in the safety assessment.

Let us begin by talking about the monitors of the system, the first one being the \textbf{RouteMonitor}. This class tracks the vehicle's progress along its route and ensures it stays within the allowed lanes. The route needed to drive is obtained from the GlobalRoutePlanner (provided by CARLA developers), which returns a list of waypoints given a list of coordinates. The coordinates come from path files discussed in \autoref{sect-5.2}. Using these waypoints, the route that is monitored by the RouteMonitor is created by recursively finding all the lanes that the vehicle could use. At each time step, the RouteMonitor checks if the car has not left the permitted lanes and tracks the number of route points that have been covered. This metric is later used to calculate route completion. At the end of the simulation, the RouteMonitor saves the route completion, whether the finish was reached, and the coordinates of all the route points and whether they have been reached to the data/recordings/participant\_[A]/scenario[1]/route\_data.xml file.

Another monitoring class implemented was the \textbf{CollisionMonitor}, which records data about collisions that the vehicle got involved in. Whenever a crash occurs and is detected by the CollisionSensor present in the Manager class, the CollisionMonitor records information about it, including the collision type (collision with a human, another vehicle, road objects, or buildings), the penalty points given to the driver, the time step, and the coordinates where it happened in the recording/recordings/participant\_[A]/scenario[1]/collision\_data.xml file.

In addition to the route and collision monitors, a \textbf{SpeedingMonitor} was developed to record data about instances when the vehicle was speeding. At each time step, the Manager class checks if the driver is exceeding the speed limit, and the SpeedingMonitor records data about it, including the allowed speed at that location, the vehicle's speed, the penalty points received for speeding, the time step, and the coordinates where it happened. This data is saved to the data/recordings/participant\_[A]/scenario[1]/speeding\_data.xml file.

A \textbf{LaneMonitor} was also implemented to record instances when the vehicle crossed solid and double solid road markings. In addition, instances of crossing broken lines without showing correct turn indicators do not go unnoticed. The LaneMonitor saves the lane type, the penalty points given, the time step, and the coordinates to the data/recordings/participant\_[A]/scenario
[1]/lane\_marking\_violation\_data.xml file.

A \textbf{TrafficMonitor} was created to track stop sign, stop marking and traffic light violations. The Manager class observes the vehicle at each time step and records traffic violations in the data/recordings/participant\_[A]/scenario[1]/road\_traffic\_violation\_data.xml file. The saved data includes the violation type, the penalty points given, the time step, and the coordinates where it happened.

Finally, a \textbf{VehicleLightMonitor} was implemented to register data about the state of the vehicle's lights and any improper use. For example, failing to use turn indicators or not having headlights in the dark would result in light violations recorded in the data/recordings/participant\_[A]/scenario[1]/vehicle\_light\_misuse\_data.xml file. The saved data included the violation type, the penalty points given, the time step when it occurred, and the location where it happened.

In addition to the monitor classes, a \textbf{Recorder} class is responsible for creating a log file following the CARLA standards that holds all the data from the simulation. The file can later be used to replay the simulation using the \textbf{Replayer} class via the replay.sh script that can be found in {software/carla\_scripts} directory. 

When the simulation ends, the manager instructs all its monitors to write the data stored in their buffers to the corresponding files. For example, participant\_A's simulation results of scenario one would be stored at data/recordings/participant\_A/scenario1/ in different .xml files, e.g. collision\_data.xml, speeding\_data.xml. The manager also gives a command to the Recorder to stop recording and save the .log file of the simulation that can be later used to replay the simulation. This gives the ability to analyse the simulations multiple times and thus catch all the minor details that can be important or monitor some data manually if needed.

To better understand how the data is stored or how the monitoring works, the reader is encouraged to look at the implemented classes in the software/carla\_scripts/recording directory in the source code. Additionally, the overview of the observer design pattern can be seen in \autoref{fig:simulation_process} in \autoref{chap:eight}. To see what the recorded data files look like, please refer to \autoref{fig:collision_data} in \autoref{chap:a} showing an example of collision\_data.xml file or explore the data/recordings directory.


\section{Metrics to record} \label{sect-6.2}
In this section, we look at what quantitative metrics from the recordings we can use to evaluate a vehicle's performance in a given scenario. After long consideration, two groups of metrics were developed: time and accuracy-related metrics allowing to give positive points to the user and safety-related metrics, which give penalty points to the user for violating some safety requirements. The two groups are discussed in the following two subsections.

\subsection{Time and accuracy-related metrics} \label{sect-6.2.1}
The first one considers how neatly and optimally the vehicle is being operated and includes these metrics:

\textbf{Timeliness} --
Each scenario has a specific optimal time to complete without violating road rules or breaking speed limits. The timeliness is calculated by dividing the optimum time value by the time the driver took to complete the route. The quicker the driver is, the larger the timeliness value will be. The value proportionally increases or decreases the driver's score depending on his swiftness.

\textbf{Proportion of route completed} --
The route is a line of dots on the map, the first being the starting position of the vehicle and the last being the finish position of the path that needs to be driven. Each dot has coordinates associated with it and is in the middle of the lane. This metric shows what percentage of waypoints were covered. For the waypoint to be considered covered, the vehicle's centre point must pass it within a certain distance. This ensures that if a car is driving not within the boundaries of the lane that it can drive on (one wheel is on the line, for instance), the waypoint will not be covered, and thus the final score will be lower. It is worth noting that if the road has multiple parallel lanes that can be driven by the vehicle, the algorithm checks if the vehicle is within the boundaries of any of those lanes. If it is not within one of those lanes but is on some of the lane markings between two lanes and is showing a turn indicator (the vehicle is changing lanes), the waypoints within a certain distance from the vehicle's centre position will be covered. This is an exception to the rule that the vehicle must stay within the lane's bounds. However, because the action takes place in a simulated world and it is difficult to feel the vehicle and be convinced that it stays within the bounds of the lane, the gamma discount factor will be applied to reduce this metric's weight.

\subsection{Safety related metrics} \label{sect-6.2.2}
The second group is evaluating how safely the vehicle is being driven and considers these metrics:

\textbf{Speeding instances} --
This metric indicates how many penalty points were assigned to the user due to exceeding speed limits. This is important because the vehicle's speed directly impacts the safety of the people sitting inside and the safety of other drivers, passengers, pedestrians, and animals. In an ideal situation, both AV and the human-drivers need to obey speed limits because speeding delays the reaction time, narrows the vision angle and increases the braking distance resulting in diminished safety.

\textbf{Collision instances} --
This metric shows how many negative points the vehicle earned throughout the simulation because of collisions with any objects (pedestrians, road objects, other vehicles, buildings). Moreover, collisions while speeding will result in even more negative points added to the final score because hitting something at high speed, will typically result in more damage done and any violation done while speeding will almost always result in the driver being guilty and responsible for causing the incident.

\textbf{Traffic light violations} --
This metric shows how many negative points the vehicle earned throughout the simulation because of running a red traffic light. In the case of a vehicle violating these rules while speeding, even more negative points will be added to the score 
because the risk of causing a severe road incident is increased, resulting in diminished overall road safety.

\textbf{Lane marking violations} --
This metric shows how many negative points the vehicle earned throughout the simulation for crossing road lines other than broken ones (- - - - -). Lane markings have to be obeyed as they are there to prevent people from overtaking or turning where it is unsafe to do. Again, lane violations while speeding will give more negative points than not speeding.

\textbf{Vehicle light violations} --
Driving at night without vehicle headlights on or in a fog without headlights and fog lights on will result in negative points given to the vehicle. In addition, changing lanes without respective turn indicators blinking will result in a worse outcome from the route metric. Showing turn indicators and having headlights on in the dark is extremely important for the vehicle's passengers and other traffic participants. The fact that all the road actors are visible and their intentions are clear directly impacts how safe the road is.

Another important metric for evaluating safety is the number of penalty points the participant receives for not stopping at stop signs and stop lane markings. This metric is of great importance as stop signs and markings are typically placed in dangerous areas. However, the current version of CARLA does not provide an efficient way to check for these violations thus this is left as a place for further improvements. More about it will be talked about in \autoref{chap:ten} when we talk about the possible future improvements of this project.









\section{Evaluating the performance} \label{sect-6.3}

Now that monitoring of the vehicle's actions was discussed and the metrics being captured were presented, it is time to give an overview of how the performance can be evaluated. Let us begin by introducing a table with penalty scores for violations of different safety requirements. This table can be found in \autoref{chap:a} in \autoref{tab:penalty_values}. It consists of three columns: the first one highlights the category of the violation, the second one gives the number of penalty points a driver is assigned if he violates the requirement, and the third column shows the values that are assigned to the driver's score for doing the violation while speeding.

Note that the values in the table are just a proposition of what could be used, and they can and should be changed to fit the project's needs accordingly. The weights shown in the table are used to assess the negative part of the driver's performance, namely how unsafe it is.
All the values introduced in this section are used to evaluate the performance of participants competing in scenarios in \autoref{chap:nine}.

In order to evaluate the driver's performance, a mathematical formula was developed that consists of three main elements: the positive score, the penalty score and the discount factor $\gamma$. The discount factor is a hyper parameter and is there to deal with the inaccuracies caused by the simulated environment. It should be adjusted depending on the simulated environment and the objective of the research. The exact purpose of the $\gamma$ will be discussed later when we will discuss how the penalty score is calculated because it is heavily impacted by $\gamma$.


Let us first talk about the positive evaluation element. It is made of the road completion $c$ multiplied by the scenario difficulty $d$ multiplied by the timeliness of the driver. The meaning of the $c$ is what proportion of the route the vehicle completed successfully. The difficulty $d$ is retrieved from the scenario file, indicating how challenging the scenario was. The timeliness is the optimal scenario completion time divided by the time it took for the driver to complete it. In simpler terms, the quicker the more complex scenario is completed, with the driver sticking to the path without any deviation, the higher the positive score they will receive.



The penalty score is the sum of all the penalty points from all monitors multiplied by the discount factor $\gamma$. It was mentioned before that $\gamma$ is a hyper-parameter, meaning that the $\gamma$ value can change the penalty score's effect on the final score. It is used to deal with the inaccuracies of the simulation and monitor malfunction. To exemplify this, consider a case where $\gamma = 1$. In that case, we are essentially saying that all the penalty scores received by the vehicle are correct and reasonable, giving the final score a maximum negative amount. However, the world is imperfect, especially the simulated world; thus, it is rational to assume that not all the sensors are 100\% correctly obtaining all the data and not all the physics are impeccably implemented. As we will see in \autoref{chap:nine}, there are many instances when simulator physics breaks down, and weird things start to happen. For example, upon colliding with another car, a car gets tossed into the air and lands a few streets further. Continuing about the monitors, the traffic monitor might capture that the vehicle did not stop at a stop sign, although it did, just a few centimetres before it. Naturally, a parameter is needed to account for the unaccountable. By using the discount parameter, we are reducing the total penalty score while at the same time also reducing the miscalculation influence on the final score. Depending on the simulator and how well the physics and monitoring work, the $\gamma$ value should be adjusted.
Summing all up, the final formula makes an equation shown below:
\begin{equation}
    score = c \times \frac{t_o}{t} \times d - \gamma \times \sum_{x=1}^{n_m} f(m_x)
\end{equation}
\begin{conditions}
\gamma & discount factor and $0 < $$\gamma$ $\leq 1$ \\
m & a set of lists of penalty values \\
 f(m_x)     &  sum of all the penalty points in list $m_x$ and $m_x \in [0; +\infty)$ \\
 c     &  proportion of route completed and $0 < c \leq 1$ \\
 t_o &  optimal amount of time (in seconds) to complete the scenario and $t_o \in (0; +\infty)$ \\
 t &  amount of time to complete the scenario and $t_o \in (0; +\infty)$ \\
 d &  difficulty of the scenario and $0 \leq d \leq 1000$ \\
 n_m &  number of lists of penalty values in set $m$ \\
\end{conditions}

However, it was not mentioned yet, how the function $f(m_x)$ is calculated. If we assume that $f(m_x) = f(x)$, since $m_x$ is a list of penalty scores of metric x, we get the formula shown below:

\begin{equation}
    f(x) = \sum_{i=1}^{n_x} x_i
\end{equation}
 \begin{conditions}
 x_i &  $i_{th}$ score from list $x$ and $x_i \in (0; +\infty)$ \\
 n_x &  number of scores in list $x$ \\
\end{conditions}

Now that the final formula has been introduced, it is time to return to one bit that was mentioned but not analysed yet, the optimal time value $t_o$.

The purpose of the $t_o$ is to proportionally increase the final positive score if the driver is driving faster than the average and proportionally decrease it if the driver is driving more slowly than the average driver. If we look back at the score calculation formula, we can see that the optimal time value $t_o$ is divided by the time the driver took to complete the route. The positive score is mainly determined by the scenario difficulty $d$, while the $c$ is there to either retain the maximum score or decrease it if the driver did not complete the whole route. On the other hand, the $t_o$ divided by $t$ increases the score if the driver is quick and decreases it if he is slow. Let us take a look at an example. Suppose the scenario's difficulty is 500 and the optimal time value to complete that scenario $t_o$ is 200 seconds. If the driver finished the route in 100 seconds, meaning he is two times quicker than the calculated average, he gets $c \times 2 \times 500 = 1000c$ positive points for the scenario. However, if he were to complete the scenario more slowly than the optimal, he would get $score < d$.

Given the simulated environment, this article proposes a mathematical formula for $t_o$ evaluation. It consists of three main parts. The first is the distance $s$ divided by the average maximum allowed speed $v_{avg}$. This part tells us the ideal time the vehicle would take to finish the distance if it was driving in a straight line without any stops and other traffic. However, the roads are full of other drivers changing lanes, slowing down, making turns, talking on the phone and not seeing the green light, etc.
For this reason, the first part of the formula is multiplied by $1 + \alpha$, where $\alpha$ denotes the intensity of traffic $\alpha = 1$ being the road is almost a gridlock and $\alpha = 0$ being the road is completely empty, and there are no other vehicles on it except the driver himself. Of course, intensity value should consider other factors such as road works, diversions or road surface quality. However, for the purpose of this project and the simulator's limitations, we are only considering the traffic density. The third part of the formula is a sum of all stops the vehicle has to make on the road. This includes pedestrian crossings, traffic lights, stop signs/markings and others. The complete formula for $t_o$ can be found below:
\begin{equation}
    t_o = \frac{s}{v_{avg}} \times (1 + \alpha) + \sum_{i=1}^n t_{avg\_ni}
\end{equation}
 \begin{conditions}
 s &  length of the route in m \\
 v_{avg} &  average speed limit over the route in m/s \\
\alpha &  the intensity value and $0 \leq \alpha \leq 1$ \\
 n &  number of of stops a vehicle has to make \\
  t_{avg\_ni} &  average time in seconds the vehicle has to be stopped for at stop $ni$ \\
\end{conditions}

However, a sharp-sighted reader might notice that an accurate calculation should consider the historical data, thus allowing for more confidence about the calculated value. Unfortunately, there is currently no source that could provide historical data for the simulated environment. Even if there were one, it would not necessarily be accurate one since the parameters of the simulator can be easily customised, affecting how the data is captured. Moreover, the $t_o$ does not include variables determining how long the vehicle would take to reduce the speed when a stop is encountered and how much time and distance it would cover to reach the speed limit again after the obstacle is passed. The formula implicitly assumes that this metric is handled by the $t_{avg\_ni}$ indicating the average time the vehicle takes to pass a probable stop. Additionally, it is worth mentioning that the last bit of the $t_o$ formula, namely the sum of all the stops, also includes the inevitable stops generated by the scenario generation algorithm. As mentioned in \autoref{chap:five} and \autoref{chap:ten}, in the future, the scenario generation algorithm could be improved to generate individual situations and not only general scenarios. If this was the case, and the algorithm had generated a situation that involves a collision, the average time the driver would take to drive past that collision should be included in the sum of all stops. This is because every detail, affecting the speed of the vehicle is significant and should be taken into account.

In summary, although the research in this project uses the introduced formula for safety score calculation, it serves only as a proposition of what could be used for evaluation. In fact, the formula used to evaluate the software in \autoref{chap:nine} is simplified because of the simulator's limitations and poor structuring of the OpenDRIVE maps used. It considers only junctions as inevitable stops that the vehicle has to make and gives them an average waiting time of 12 seconds. This is because of the missing information in the OpenDRIVE map files the CARLA development team provided. An example of this can be seen from \autoref{fig:speed_yes} until \autoref{fig:town07_speed}, in \autoref{chap:c} that exemplify how road elements in the same map are specified differently and that on average, less than 30\% of the road elements have a speed limit specified. In addition, some maps contain information about where the traffic lights are, and others do not include that information. Much more research and proving needs to happen before we can establish the correctness and quality of the safety evaluation function. A significant amount of effort was paid in search of an existing function in the literature that would have already survived the test of time and could be used here. However, no function found in the literature could suit the needs of this project. For this reason, the article aims to be inventive but risk-taking simultaneously, leaving the reader to judge the formula's correctness.
