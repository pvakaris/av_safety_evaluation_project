\chapter{Human drivers and the road} \label{chap:three}

This chapter discusses how people behave on the roads and why collecting real-life traffic data is crucial for AI model training. \autoref{sect-3.1} analyses traffic incident data collected in the EU and U.S. in 2020, while the following \autoref{sect-3.2} links data to human behaviour and analyses why some types of accidents happen more frequently than others. In \autoref{sect-3.3}, we finish the chapter by discussing how the traffic data and human psychology analysis can help us understand how the simulated scenarios should be designed and what situations they should reenact. Let us begin by looking at the statistics.

\section{Road incident statistics} \label{sect-3.1}
For the purpose of this research, it is important to understand the situations that lead to human driver failures on the road. These failures can occur due to a variety of factors, including distractions, participant impairment, road and weather conditions, place.

According to the European Commission’s published data for road safety in the EU in 2020 \cite{eu_commision}, 52\% of fatal road incidents happened in rural areas, while 40\% occurred in urban areas and only 8\% on motorways. The analysis also indicates that 70\% of all fatalities in urban areas were made up of vulnerable road users such as cyclists, moped and motorcycle drivers and pedestrians, that account for the vast 37\% of victims in city areas.

In the traffic safety facts annual report tables provided by The National Highway Traffic Safety Administration of U.S. Department of Transportation \cite{nhtsa_data}, it is clear that the majority of fatal road crashes in the U.S. in 2020 happened from 3PM until 11:59PM on working days and from 6PM and 3AM on weekends. The NHTSA's data set also indicates that the majority (83\%) of fatal crashes happened during normal weather conditions while rainy weather caused only 7\% of fatal incidents. Out of the 83\% of crashes that happened during normal weather conditions, 46\% of accidents happened during daylight while 28\% happened at night and 21\% happened at night but in lit areas.

This shows an interesting fact that although there is a general perception that driving in snowy or foggy conditions is more prone to becoming a victim of an accident, the statistics show that it is entirely contrary. Most crashes happen when the conditions are best for all the human drivers (well-lighted areas in the middle of the day when everything is well observable).

It is also worth mentioning that 30\% of all road fatalities occurred in alcohol-impaired road accidents. Before linking the data covered in this section to the goal of this project, it is important to look at why we have such a large number of incidents on the road.

\section{Human nature and behaviour} \label{sect-3.2}
A standard individual essentially has four main sensors when it comes to driving (two eyes and two ears) that provide information about the surrounding world and a brain which is responsible for making decisions.

Although a people have everything necessary to operate the vehicles on roads and following the established rules at all times they could do that successfully and safely, there are many factors that could make the roads unsafe regardless.

One reason is that emotions are a key aspect of human life and can heavily impact how a person makes decisions. When a person is feeling strong emotions such as anger, frustration, or stress, it can affect their ability to concentrate and make good decisions while driving.
For example, a person who is feeling angry may be more likely to take risks or engage in aggressive driving behaviors, such as tailgating, speeding, or cutting off other drivers. This can increase the risk of collisions and accidents on the road.

Similarly, a person who is feeling stressed or distracted may have difficulty paying attention to the surroundings or react to situations. This is due to the humans' limited cognitive energy reserve. Neurological science has demonstrated that the human brain is incapable of focusing on two things at once as stated in the article ``The Impossibility of Focusing on Two Things at Once'' written by M. Hernandez \cite{focusing_on_things}. This means that even listening to music or having a passenger telling a cunning story about his recent promotion reduces the driver's alertness and clouds the perception of their surroundings.

There are other factors that can have significant impact on how a person is able to operate the vehicle one being fatigue or impairment, such as being influenced by alcohol or drugs.

Artificial intelligence has the potential to improve driving safety by addressing some of these factors. On one hand, AVs are essentially exceptionally sophisticated software systems with dozens and possibly hundreds of sensors providing reasonably accurate and reliable information about the surroundings and multiple cores for computation able to make trillions of calculations every second. Moreover, systems controlling the vehicle cannot be distracted by either delicious food advertisements on the street or a song on the radio. Even in the case of the passenger being able to communicate with the vehicle vocally, it could sacrifice some of the computational power of one of its multiple cores for that fraction of a second needed to calculate the most appropriate answer. The AV systems cannot become fatigued or get influenced by substances or emotions. If programmed well their behavior and decision making cannot be altered by external individuals or systems.

However, it is important to note that AI systems are not immune to hardware malfunctioning, software errors or cyber attacks by very sophisticated attackers. AV safety was discussed in \autoref{sect-2.4}.

\section{Statistics relevance to the research} \label{sect-3.3}
Now that we discussed the trends in road accident statistics and looked a bit into the differences between human and AI vehicle operators, it is time to understand why all this information is crucial for developing adequate systems able to generate drivable scenarios and analyse vehicle safety in situations that are the most prone to accidents.

Comprehensive road statistics analysis can help ensure that the simulated scenarios accurately reflect the real-world conditions that AVs will encounter. Scenario generation has to take into account when the most accidents occurred and what factors, such as weather, lighting, road surfaces, or places, influenced them the most. Extracting accurate data from the statistics allows testing the AVs in realistically challenging situations and shows that the research is as representative of the real world as possible.

One way road data could be used is to train AI models to predict what situations could be challenging to drive and prone to accidents. We will discuss more about scenario generation when we introduce a way of automatically generating driving scenarios using a machine learning model in \autoref{chap:five}. The proposed ML algorithm is able to predict the parameters of new scenarios based on the scenario examples it has seen previously and then put the numbers to life with the help of the CARLA modules.

From the data, analysed in \autoref{sect-3.1}, it is clear that most simulated scenarios should occur in rural and urban areas reflecting situations that impact road safety the most. However, highway roads also account for a substantial proportion of road incidents; thus, some scenarios should also happen on highways.

Regarding the weather statistics from \autoref{sect-3.1}, it is clear that to mirror the conditions prone to vehicle collisions; scenarios need to be designed in various weather conditions, with most of them based on normal weather circumstances.

The datasheet also indicates that many road incidents involved pedestrians and two-wheel vehicle drivers, e.g., cyclists and motorcycle drivers. 

To sum up, it is clear that in order to develop a reliable system that can generate driving scenarios that accurately reflect real-world conditions, an enormous amount of diverse traffic data is needed that could be restructured so it can be used to train machine learning algorithms. More details about this process and the chosen way to implement the framework component responsible for scenario generation are discussed in \autoref{chap:five}.
